\documentclass[a4paper,10pt,makeidx]{article}
\usepackage{epsfig}
\usepackage[bahasa]{babel}
\usepackage{picinpar}
\title{{\HUGE Latihan Soal \\ Olimpiade Komputer}}
\author{dwi sakethi, http://dwijim.staff.unila.ac.id}
\makeindex
\begin{document}
\begin{center}
 \thispagestyle{empty}
 \newcommand{\HRule}{\rule{\linewidth}{1mm}}
 \setlength{\parindent}{0mm}
 \setlength{\parskip}{0mm}
 \pagebreak
 %  \vspace*{\stretch{1}}
   \HRule
   \begin{flushright}
      \Large Latihan Soal      \\[5mm]
      \Huge Olimpiade Komputer \\[5mm]
    \end{flushright}
    \HRule

      \vspace*{\stretch{1}}

    \begin{flushright}
      \Large Oleh : \\[5mm]
      \Large dwi sakethi, s.si, m.kom      \\[2mm]
      \small http://dwijim.staff.unila.ac.id \\[2mm]
             dwijim@unila.ac.id     \\[2mm]
             0816 403 432                  \\[2mm]
             pengrajin teknologi informasi \\[5mm]
      \vspace*{\stretch{1.5}}
      \Large \textsc{Lembah (jangan) Banjir} \\[5mm]
      \Large \textsc{Tanjung Seneng 2006} \\[5mm]
    \end{flushright}
\end{center}
\newpage
\pagenumbering{roman}
\section{Kata Pengantar}
\par
\indent
\indent
\index{Allah swt.}
\index{Rasulullah saw.}
\index{sholawat}
\index{{\itshape download}}
Puji yang sejati hanya untuk {\itshape Allah} Yang Maha Tinggi, puja yang sempurna
hanya untuk {\itshape Allah} Yang Maha Kuasa. {\itshape Sholawat} dan {\itshape
salam} semoga senantiasa dilimpahkan kepada tauladan semua yang mengaku
merupakan himpunan bagian dari kelompok manusia,
Muhammad saw., keluarganya, sahabatnya, pengikutnya dan seluruh muslimin kapan
dan dimana {\itshape bae}.
\par
\indent
\indent
Tulisan ini disusun sebagai rangkaian kegiatan ketika membina tim olimpiade
komputer yang dimulai dengan pembinaan terhadap SMA Negeri 1 Gading Rejo
Lampung Tengah. Untuk itu saya sangat berterima kasih kepada Bapak Jumiran
dan anak-anak SMA Negeri 1 Gading Rejo waktu itu, yang mohon maaf sekarang
saya sudah lupa lagi ... :-) Kegiatan itu kalau saya tidak salah dilakukan 
tahun 2004. Kemudian berlanjut membina tim olimpiade komputer SMA Negeri 2
Bandar Lampung pada bulan Agustus 2006.
\par
\indent
\indent
Semoga ini menjadi bagian amal kebaikan yang dapat mendorong kesejahteraan
di bumi pertiwi ini ... Sehingga rakyat selalu mendendangkan lagu {\bfseries
Disini senang, di sana senang}, bukan lagu {\bfseries Padamu Negeri}.\\ \\ \\
dwi sakethi \\
dwijim@unila.ac.id \\
http://dwijim.staff.unila.ac.id \\
hp : 0816 403 432 (nomor cantik ya ...) \\
pengrajin teknologi informasi \\
\newpage
\tableofcontents
%\listoftables
%\listoffigures
\setcounter{page}{1}
\pagenumbering{arabic}
\newpage
\section{Reverse}
\par
\indent
\indent
{\itshape Wah} ... mudah amat ya ... soal olimpiade komputer. Yah ... ini 
memang soal pemanasan dan pengenalan, jadi sangat mudah. Soalnya sebagai
berikut : \\
\par
\indent
\indent
Buatlah program REVERSE.PAS menurut penjelasan berikut ini. Sebagai latihan,
Anda belajar menentukan cara bagaimana membaca masukan string yang sangat
panjang.
\subsection{Masukan}
\par
\indent
\indent
Progran itu harus membaca masukan dari file bernama REVERSE.IN. File ini akan
berisikan satu baris teks dengan $panjang\le 1000$ karakter.( {\itshape Hint}
: ini lebih panjang dari panjang maksimum string di Pascal).
\subsection{Keluaran}
\par
\indent
\indent
Program harus menuliskan keluaran dalam file REVERSE.OUT.Keluaran adalah hanya
satu baris teks yaitu string hasil pembalikan yang telah anda lakukan dari string
masukan. \\
reverse.in         \\
Olimpiade Nasional \\

reverse.out        \\
lanoisaN edaipmilO \\
\subsection{Contoh Penyelesaian}
Penyelesaian : Berikut ini salah satu contoh penyelesaian masalah di atas.
Masih banyak cara yang bisa dilakukan untuk menyelesaiakannya.
\begin{verbatim}
program membalik_deretan_karakter;
uses crt;
var i,jumlah  : integer;
    file_data : text;
    karakter  : char;
    data_asal : array[1..1000] of char;
begin
   clrscr;
   jumlah := 0;
   assign(file_data,'data.txt');
   reset(file_data);
   writeln('data reverse');
   writeln('--------------------------------------------------');
   while not eof (file_data) do
      begin
        jumlah := jumlah + 1;
        read(file_data,karakter);
        write(karakter);
        data_asal[jumlah] := karakter;
      end;
   close(file_data);
   writeln;
   writeln('--------------------------------------------------');
   writeln('hasil reverse');
   writeln('--------------------------------------------------');
   for i:= jumlah downto 1 do write(data_asal[i]);
   readln;
end.
\end{verbatim}
\subsection{Contoh hasilnya}
\begin{verbatim}
data reverse
--------------------------------------------------
program membalik_deretan_karakter; salam keadilan dan kesejahteraan untuk rakyat
 di bumi pertiwi. supaya yakin maka tulisan ini dibuat lebih dari 255 karakter.
apakah anda sudah mempersiapkan bekal untuk kehidupan setelah kematian nanti ? d
wi sakethi pengrajin teknologi informasi 0816 403 432

--------------------------------------------------
hasil reverse
--------------------------------------------------

234 304 6180 isamrofni igolonket nijargnep ihtekas iwd ? itnan naitamek haletes
napudihek kutnu lakeb nakpaisrepmem hadus adna hakapa .retkarak 552 irad hibel t
aubid ini nasilut akam nikay ayapus .iwitrep imub id taykar kutnu naarethajesek
nad nalidaek malas ;retkarak_natered_kilabmem margorp
\end{verbatim}

\section{Menghitung Massa}
\par
\indent
\indent
Suatu molekul terdiri atas sejumlah atom dan tersusun membentuk rumus kimia yang
dituliskan dengan huruf-huruf yang menyatakan masing-masing atom ini. Misalnya
H menyatakan atom hidrogen, C menyatakan atom karbon, O menyatakan atom
oksigen. Jadi rumus kimia COOH menyatakan suatu molekul yang berisikan satu
atom karbon, dua atom oksigen dan satu atom hidrogen.
\par
\indent
\indent
Untuk menuliskan rumus ini secara efisien  kita menggunakan aturan-aturan
berikut ini.
\par
\indent
\indent
Huruf-huruf yang menyatakan beberapa atom dapat dikelompokkan dengan pembatas
tanda kurung yang disebut juga dengan istilah gugus atom. Misalnya rumus CH(OH)
berisi gugus OH. Dalam suatu gugus bisa terdapat gugus-gugus lebih kecil. Untuk
menyederhanakan suatu rumus kimia, kemunculan sejumlah huruf secara
berturut-turut dapat digantikan dengan satu huruf saja tapi diikuti oleh suatu
bilangan yang menyebutkan jumlah kemunculannya. Misalnya huruf COOHHH dapat
ditulis sebagai CO2H3 dan ia mempresentasikan suatu molekul yang berisikan
satu atom karbon, dua atom oksigen dan tiga atom hidrogen.
\par
\indent
\indent
Selanjutnya, kemunculanya yang berturut-turut dari gugus yang sama  dapat
digantikan  dengan gugus tersebut diikuti oleh bilangan yang menyatakan jumlah
kemunculan gugus tersebut. Misalnya CH (CO2H)(CO2H)(CO2H) dapat dituliskan
sebagai CH(CO2H)3 dan molekul tersebut berisikan empat atom karbon, dua atom
oksigen dan tiga atom hidrogen.
\par
\indent
\indent
Dalam rumus kimia sebenarnya tentu bilangan yang menyatakan pengulangan
kemunculan  suatu\/gugus  atom tersebut bisa berharga berapapun asal $\ge 1$.
Dalam soal di sini bilangan tersebut  dibatasi sampai dengan 9.
\par
\indent
\indent
Massa dari suatu molekul adalah jumlah dari massa  dari setiap atom yang
tergantung di dalamnya. Satu atom hidrogen memiliki massa satu, satu atom
karbon memiliki  massa 12 dan satu atom oksigen memiliki  massa 16.
\par
\indent
\indent
Tuliskan suatu program  dengan nama MASSA.PAS yang dapat  manghitung massa
molekul dari rumus molekul yang diberikan.
\par
\indent
\indent
\subsection{Format Masukan}
\par
\indent
\indent
File mmasukan adalah file teks dengan nama MASSA.IN. File berisi satu baris
yang didalamnya tertuliskan rumuus molekul yang hendak dihitung massanya.Rumus
molekul hanya akan berisikan kemungkinan karakter-karakter H, C, O, (,), 2,
3, ..., 9. Panjang string tidak akan lebih dari 100 karakter.
\par
\indent
\indent
\subsection{Format Keluaran}
\par
\indent
\indent
File keluaran adalah file teks dengan nama MASSA.OUT. Satu-satunya baris keluaran
hanya berisikan massa dari molekul yang dinyatakan dengan rumus yang diberikan.
Bilangan massa tidak akan akan lebih besar dari 1000 karakter.
\par
\indent
\indent
\subsection{Contoh penyelesain}
\par
\indent
\indent
\begin{verbatim}
uses crt;
var hasil_kurung_total,hasil_kurung,hasil_akhir,kimia : string;
    p1,p : string;
    massa,tingkatx,k,i,j,jumlah_kurung   : integer;
    akhir_sebelum,awal_sebelum,panjang,tingkat : integer;
    jumlah_asal,sekarang,kode_error : integer;
    awal,akhir,tingkat_kurung_awal,sudah_dihitung : array[1..25] of integer;
    pengali,kurung_punya,tingkat_kurung_akhir,kepakai : array[1..25] of integer;
    ada_kurung,selesai : boolean;
begin
   clrscr;
{ ----------------------------------------------------------
  silahkan ganti-ganti bentuk kimia sesuai yang dikehendaki
  ---------------------------------------------------------- }
   kimia :='O(CO2H)3';
   kimia :='CH(CO2H)3';
   kimia :='((CH)2(OH2H)(C(H))O)3';
   kimia :='COOH';
   writeln('bentuk asal : ',kimia);
   hasil_akhir := '';
   panjang := length(kimia);
   tingkat := 0;
   jumlah_kurung := 0;
   for i:=1 to panjang do 
     begin
       p := copy(kimia,i,1);
       kepakai[i] := 0;
       sudah_dihitung[i] := 0;
       tingkat_kurung_akhir[i]:=0;
       if p='(' then
          begin
             inc(tingkat);
             inc(jumlah_kurung);
             awal[jumlah_kurung] := i;
             tingkat_kurung_awal[jumlah_kurung] := tingkat;
             akhir[jumlah_kurung] := 0;
             kurung_punya[i] := jumlah_kurung;
          end
      else
        if p=')' then 
            begin
                kurung_punya[i] :=0;
                tingkat_kurung_akhir[i] := tingkat_kurung_awal[jumlah_kurung];
                akhir[jumlah_kurung] := i;
                dec(tingkat);
            end;
     end;
{ mencari posisi kurung tutup }
     tingkatx := 0;
     jumlah_kurung := 0;
     for i:=1 to panjang do
         begin
           p := copy(kimia,i,1);
       if p='(' then
          begin
             inc(tingkatx);
             inc(jumlah_kurung);
          end
       else
           if p=')' then
              begin
                 for j:=i downto 1 do
                     begin
                        p1 := copy(kimia,j,1);
                        if p1='(' then
                           begin
                              if kepakai[j]=0 then
                                 begin
                                    akhir[kurung_punya[j]] :=i;
                                    kepakai[j] := 1;
                                    j := 1;
                                 end;
                           end;
                     end;
                  dec(tingkatx);
              end;
         end;
for i:=1 to jumlah_kurung do
    begin
       write('tingkat : ',tingkat_kurung_awal[i],
               ' mulai ',awal[i],' sampai ',akhir[i]);
       p:=copy(kimia,akhir[i]+1,1);   { mencari faktor pengali untuk }
       val(p,k,kode_error);           { masing-masing kurung         }
       if (k=0) then k:=1;
       pengali[i] :=k;
       writeln(' : ',pengali[i]);
    end;
for i:=1 to jumlah_kurung do kepakai[i]:=0;
{ kepakai di sini digunakan untuk menandai bahwa
  suatu karakter sudah dihitung }

selesai := false;
   hasil_kurung_total := '';
repeat
   sekarang := 0;
   for i:=1 to jumlah_kurung do
     begin
        if (tingkat_kurung_awal[i]>sekarang) and (kepakai[i]=0)
           then sekarang:=i;
           { akan memproses tingkat kurung tertinggi dan
             pada posisi itu memang belum diproses }

     end;
   write('kurung ke-',kurung_punya[awal[sekarang]],' ');
   write('tingkat ',tingkat_kurung_awal[sekarang],' : ');
          kepakai[sekarang]:=1;   { menandai bahwa posisi ini sudah
                                    diproses }
   hasil_kurung := '';
   for i:=awal[sekarang] to akhir[sekarang] do
       begin
          p := copy(kimia,i,1);
          if ((p='C') or (p='H') or (p='O')) and (sudah_dihitung[i]=0) then
             begin
                sudah_dihitung[i]:=1;
                p1 := copy(kimia,i+1,1);
                val(p1,k,kode_error);
                if (k>=2) then
                    for j:=1 to k do hasil_kurung:=hasil_kurung + p
                else
                    hasil_kurung := hasil_kurung + p;

             end;
       end;

       { mencari apakah di dalam kurung yang sekarang
         ada kurung lagi di dalamnya }
         ada_kurung := false;
            for i:=awal[sekarang]-1 downto 1 do
              begin
                 p:=copy(kimia,i,1);
                 if (p='(') then
                    begin
                       awal_sebelum  := awal[kurung_punya[i]];
                       akhir_sebelum := akhir[kurung_punya[i]];
                       if (awal_sebelum<awal[sekarang]) then
                          if (akhir_sebelum>akhir[sekarang]) then
                             begin
                                write('awal : ',awal_sebelum,' - ',awal[sekarang],' ');
                                write('ada di dalam kurung',' ');
                                ada_kurung := true;
                                i := 1;
                             end;
                    end;
              end;
       { hasil yang didapat, dikalikan dengan pengali untuk
         masing-masing kurung }
         k := pengali[kurung_punya[awal[sekarang]]];
         hasil_kurung_total := '';
         if (awal[sekarang]<>1) and (akhir[sekarang]<>panjang-1) then
            for i:=1 to k do
                begin
                  hasil_kurung_total:= hasil_kurung_total+ hasil_kurung;
                end
         else
                hasil_kurung_total:= hasil_kurung_total+ hasil_kurung;
         writeln('pengali : ',k);
         writeln('hasil kurung : ',hasil_kurung_total);
         hasil_akhir := hasil_akhir + hasil_kurung_total;
         writeln('hasil akhir: ',hasil_akhir);
   sekarang := 0;

   { kalau semua posisi sudah diproses artinya kepakai[i]=1
     berarti proses selesai }
   selesai := true;
   for i:=1 to jumlah_kurung do
       if kepakai[i]=0 then
          begin
             selesai := false;
             i := jumlah_kurung;
          end;
until selesai;

panjang := length(kimia);
hasil_kurung_total := hasil_akhir;
if not ada_kurung then
   hasil_akhir := ''
else
   hasil_akhir := hasil_akhir;

{ untuk antisipasi bentuk khusus dimana
  kurung terakhir di kolom terakhir-1, tapi
  kurung awalnya di kolom 1 }
k := pengali[tingkat_kurung_awal[1]];
if (awal[1]=1) and (akhir[1]=panjang-1) then
   for i:=1 to k do hasil_akhir := hasil_akhir + hasil_kurung_total;

{ untuk antisipasi bentuk khusus dimana
  kurung terakhir di kolom terakhir-1, tapi
  kurung awalnya bukan di kolom 1 }

if (awal[1]<>1) and (akhir[1]=panjang-1) then
   for i:=1 to k do hasil_akhir := hasil_akhir + hasil_kurung_total;

{
  kalau-kalau ada yang belum dihitung,
  artinya nilai sudah_dihitung=0
  ini terjadi kalau tidak ada kurung sama sekali }
for i:=1 to awal[1] do
    if sudah_dihitung[i]=0 then
       begin
         p := copy(kimia,i,1);
         if (p='C') or (p='H') or (p='O') then
            hasil_akhir := hasil_akhir + p;
       end;

writeln('hasil akhirnya : ',hasil_akhir);
panjang := length(hasil_akhir);
massa := 0;
for i:=1 to panjang do
    begin
       p := copy(hasil_akhir,i,1);
       if p='H' then
          massa:=massa+1
       else
          if p='C' then
             massa:=massa+12
          else
             massa:=massa+16;
    end;
writeln('Massa : ',massa);
readln;
end.
\end{verbatim}

\subsection{Contoh hasil {\itshape running}}
\begin{verbatim}
bentuk asal : COOH
kurung ke-0 tingkat -28666 : pengali : 1280
hasil kurung :
hasil akhir:
hasil akhirnya : COOH
Massa : 45

bentuk asal : CH(CO2H)3
tingkat : 1 mulai 3 sampai 8 : 3
kurung ke-1 tingkat 1 : pengali : 3
hasil kurung : COOH
hasil akhir: COOH
hasil akhirnya : COOHCOOHCOOHCH
Massa : 148

bentuk asal : ((CH)2(OH2H)(C(H))O)3
tingkat : 1 mulai 1 sampai 20 : 3
tingkat : 2 mulai 2 sampai 5 : 2
tingkat : 2 mulai 7 sampai 12 : 1
tingkat : 2 mulai 13 sampai 18 : 1
tingkat : 3 mulai 15 sampai 17 : 1
kurung ke-5 tingkat 3 : awal : 13 - 15 ada di dalam kurung pengali : 1
hasil kurung : H
hasil akhir: H
kurung ke-2 tingkat 2 : awal : 1 - 2 ada di dalam kurung pengali : 2
hasil kurung : CHCH
hasil akhir: HCHCH
kurung ke-3 tingkat 2 : awal : 1 - 7 ada di dalam kurung pengali : 1
hasil kurung : OHHH
hasil akhir: HCHCHOHHH
kurung ke-4 tingkat 2 : awal : 1 - 13 ada di dalam kurung pengali : 1
hasil kurung : C
hasil akhir: HCHCHOHHHC
kurung ke-1 tingkat 1 : pengali : 3
hasil kurung : O
hasil akhir: HCHCHOHHHCO
hasil akhirnya : HCHCHOHHHCOHCHCHOHHHCOHCHCHOHHHCO
Massa : 222
\end{verbatim}
Tapi mohon maaf sebesar-besarnya ... ternyata program tersebut belum
dapat menyelesaikan :-) ... bentuk seperti ini misalnya :
\begin{verbatim}
bentuk asal : CO2H
kurung ke-0 tingkat -28666 : pengali : 1280
hasil kurung :
hasil akhir:
hasil akhirnya : COH
Massa : 29
\end{verbatim}
Mengapa demikian ? Ya ... ini menjadi PR lanjutan untuk Anda yang
masih penasaran ... :-)

\section{Ekspresi Aljabar}
\par
\indent
\indent
Buatlah program EKSPRESI.PAS sebagai latihan Anda menjelang ON.
Latihan ini mulai agak sulit. Tujuan latihan ini untuk Anda membiasakan diri
dengan kompiler Free Pascal yang digunakan di web server saat menguji
perkerjaan
Anda yang mungkin berbeda dengan kompiler yang sering Anda gunakan selama ini.
Selain itu anda mulai berlatih pemrograman dengan tingkat kesulitan mulai
mendekati soal-soal di ON nanti.
\par
\indent
\indent
Program anda harus dapat membaca string masukan yang berisi ekspresi
aritmetika yang terdiri atas operator pangkat-kali-bagi-tambah-kurang dan
menuliskan urutan pengerjaannya yang benar. Misalnya : \\
\begin{center}
       $a-b+c/d*e/f\wedge g-h*j$
\end{center}
\par
\indent
\indent
Untuk menentukan urutan pengerjaannya dalam penulisannya operator-operator
tersebut diberikan tingkat prioritas; pangkat paling tinggi, kemudian kali
dan bagi pada prioritas
yang sama, dan terakhir tambah dan kurang, pada prioritas yang sama.
( Note :
Dalam latihan ini tanda kurung atau operator lain belum diikutsertakan).
Dengan adanya tingkat prioritas ini maka $f\wedge g$ harus dikerjakan
sebelum {\bfseries $e/f$}
atau {\bfseries $g-h$}. Jika prioritas sama sehingga mana yang di sebelah kiri
akan dikerjakan
lebih dahulu dari yang di sebelah kanan. Untuk contoh di atas
{\bfseries $c/d$}  dikerjakan
terlebih dahulu dari pada {\bfseries  $d*e$}. Dengan menggunakan nama variabel
sementara xi
untuk menerima hasil pengerjaan suatu operasi, maka salah satu urutan
pengerjaan ekspresi tersebut adalah : \\
$xl = a-b  $ \\
$x2 = c/d  $ \\
$x3 = x2*e $ \\
$x4 = f \wedge g$ \\
$x5 = x3/x4$ \\
$x6 = x1+x5$ \\
$x7 = h*j  $ \\
$x8 = x6-x7$ \\

\subsection{Masukan}
\par
\indent
\indent
Program itu harus membaca masukan dari file bernama EKSPRESI.IN. File ini akan
berisikan satu baris teks ekspresi aritmetika dengan panjang $< 256$ karakter.
Operator pangkat ditulis dengan simbol ' $\wedge$ ', operator kali dengan simbol '*',
operator bagi dengan simbol '$/$', operator tambah dengan simbol '+', dan operator
kurang dengan simbol '-'.Operand-operand-nya sendiri adalah menggunakan
karakter huruf tunggal (a-Z, A-Z) untuk memudahkan anda membaca masukan. Dalam
ekspresi tidak ada karakter spasi atau karakter lainnya selain huruf atau
karakter simbol operator tersebut di atas.
\subsection {Keluaran}
\par
\indent
\indent
Program harus menuliskan keluaran dalam file bernama EKSPRESI.OUT.Keluaran
berisikan baris-baris operasi untuk mengerjakan ekspresi masukan yang dibantu
oleh variabel-variabel sementara $x_i$. Agar keluaran menjadi unik maka urutan
sedapat mungkin dari kiri ke kanan ekspresi kecuali kalau terkait dengan
prioritas. Misalnya {\bfseries $a-b$} harus ditulis lebih dahulu
dari {\bfseries $c/d$} karena {\bfseries $a-b$} tidak
bergantung hasil {\bfseries $c/d$}. Variabel-variabel sementara $x_i$
dituliskan
sebagai karakter x dan bilangan i dengan i membesar dari baris pertama ke baris
terakhir. \\

Contoh 1      \\

EKSPRESI.IN   \\ \\
$a-b+c/d$     \\

FIle.OUT        \\ \\
$x1 = a-b   $   \\
$x2 = c/d   $   \\
$x3 = x1+x2 $   \\

Contoh 2      \\

EKSPRESI.IN   \\ \\
$c/d*e/f\wedge g$   \\


EKSPRESI.OUT     \\ \\
$x1 = c/d   $    \\
$x2 = x1*e  $    \\
$x3 = f\wedge g   $    \\
$x4 = x2/x3 $    \\

Contoh 3     \\

EKSPRESI.IN  \\ \\
$a-b+c/d*e/f\wedge g-h*j  $\\

EKSPRESI.OUT       \\ \\
$x1 = a-b       $      \\
$x2 = c/d       $      \\
$x3 = x2*e      $      \\
$x4 = f\wedge g       $      \\
$x5 = x3/x4     $      \\
$x6 = x1+x2     $      \\
$x7 = h*j       $      \\
$x8 = x6-x7     $      \\

\subsection{Contoh Penyelesaian}
\par
\indent
\indent
Berikut ini adalah contoh program untuk menyelesaikan masalah di atas.
\begin{verbatim}
program analisa_ekpresi_aljabar;
{ versi sabtu }
uses crt;
var ekspresi : string;
    batas_kiri,batas_kanan,proses,sekarang,i,j,jumlah_tanda,panjang : byte;
    tanda,karakter : array[1..255] of string[1];
    str_temp,suku_kiri,suku_kanan: string[2];
    prioritas : array[1..255] of byte;
    cari_prioritas, selesai : boolean;
    substitusi : string[6];
    jumlah_tanda_asli,nilai: byte;
    kode_error : integer;

begin
   clrscr;
   ekspresi := 'c/d*e/f^g';
   ekspresi := 'a-b+c/d*e/f^g-h*j';
   panjang  := length(ekspresi);
   jumlah_tanda := 0;
   proses := 0;

{
  proses mencari jumlah operator dan operator apa saja
  yang ada beserta tingkatnya                           }

   for i:=1 to panjang do
       begin
          karakter[i] := copy(ekspresi,i,1);
          if (karakter[i]='-') or (karakter[i]='+') then
             begin
                inc(jumlah_tanda);
                prioritas[jumlah_tanda] := 1;
                tanda[jumlah_tanda] := karakter[i];
             end
          else
             if (karakter[i]='/') or (karakter[i]='*') then
                begin
                   inc(jumlah_tanda);
                   prioritas[jumlah_tanda] := 2;
                   tanda[jumlah_tanda] := karakter[i];
                end
             else
               if (karakter[i]='^') then
                begin
                   inc(jumlah_tanda);
                   prioritas[jumlah_tanda] := 3;
                   tanda[jumlah_tanda] := karakter[i];
                end;
       end;
       jumlah_tanda_asli := jumlah_tanda;

       { mencari operator mana yang akan dikerjakan terlebih dahulu }
       cari_prioritas := false;
       repeat
          sekarang := 1;
          for i:=1 to jumlah_tanda do
             begin
                if prioritas[sekarang+1]>prioritas[sekarang] then
                   sekarang := sekarang+1
                else
                   cari_prioritas:=true;
             end;
       until cari_prioritas;
       selesai := false;
repeat
       inc(proses);
       writeln('ekspresi : ',ekspresi);
       write('tanda ke : ',sekarang,' yang mau dikerjakan ');
       textcolor(yellow+blink);writeln(tanda[sekarang]);
       textcolor(white);
       batas_kiri  := 2*sekarang-1;
       batas_kanan := 2*sekarang+1;
       writeln('batas kiri : ',batas_kiri,' batas kanan : ',batas_kanan);
       writeln('karakter batas kiri : ',karakter[batas_kiri],' ',
               'karakter batas kanan : ',karakter[batas_kanan]);
       suku_kiri  := karakter[batas_kiri];
       { kalau suku kiri=1 ini artinya x1,
         kalau suku kiri=2 ini artinya x2, dan seterusnya }
       val(suku_kiri,nilai,kode_error);
       if (nilai>0) then suku_kiri:='x'+suku_kiri;

       suku_kanan := karakter[batas_kanan];
       val(suku_kanan,nilai,kode_error);
       if (nilai>0) then suku_kanan:='x'+suku_kanan;

       substitusi := suku_kiri+tanda[sekarang]+suku_kanan;

       str(proses,str_temp);
           writeln('--------- substitusi x',str_temp,'=',
                   substitusi,' -----------');
       ekspresi := '';
       for i:=1 to batas_kiri-1 do ekspresi:=ekspresi+karakter[i];
       ekspresi := ekspresi + str_temp;
       for i:=batas_kanan+1 to panjang do ekspresi:=ekspresi+karakter[i];
       writeln('ekspresi baru setelah direduksi : ',ekspresi);

{ proses seperti di awal kembali }
{
  proses mencari jumlah operator dan operator apa saja
  yang ada beserta tingkatnya                           }
   panjang  := length(ekspresi);
   jumlah_tanda := 0;

   for i:=1 to panjang do
       begin
          karakter[i] := copy(ekspresi,i,1);
          if (karakter[i]='-') or (karakter[i]='+') then
             begin
                inc(jumlah_tanda);
                prioritas[jumlah_tanda] := 1;
                tanda[jumlah_tanda] := karakter[i];
             end
          else
             if (karakter[i]='/') or (karakter[i]='*') then
                begin
                   inc(jumlah_tanda);
                   prioritas[jumlah_tanda] := 2;
                   tanda[jumlah_tanda] := karakter[i];
                end
             else
               if (karakter[i]='^') then
                begin
                   inc(jumlah_tanda);
                   prioritas[jumlah_tanda] := 3;
                   tanda[jumlah_tanda] := karakter[i];
                end;
       end;

       { mencari operator mana yang akan dikerjakan terlebih dahulu }
       cari_prioritas := false;
       writeln('jumlah tanda ',jumlah_tanda);
       repeat
          sekarang := 1;
          for i:=1 to jumlah_tanda do
             begin
                if prioritas[sekarang+1]>prioritas[sekarang] then
                   sekarang := sekarang+1
                else
                   cari_prioritas:=true;
             end;
             if jumlah_tanda=1 then cari_prioritas:=true;
       until cari_prioritas;
       selesai := false;
       if jumlah_tanda=1 then
          begin
              str(jumlah_tanda_asli,str_temp);
              writeln('--------- x',str_temp,'=x',karakter[1],karakter[2],'x',karakter[3]);
              if jumlah_tanda=1 then selesai:=true;
          end;
       readln;
until selesai;
end.
\end{verbatim}
\subsection{Contoh hasil {\itshape running}}
\begin{verbatim}
ekspresi : a-b+c/d*e/f^g-h*j
tanda ke : 1 yang mau dikerjakan -
batas kiri : 1 batas kanan : 3
karakter batas kiri : a karakter batas kanan : b
--------- substitusi x1=a-b -----------
ekspresi baru setelah direduksi : 1+c/d*e/f^g-h*j
jumlah tanda 7

ekspresi : 1+c/d*e/f^g-h*j
tanda ke : 2 yang mau dikerjakan /
batas kiri : 3 batas kanan : 5
karakter batas kiri : c karakter batas kanan : d
--------- substitusi x2=c/d -----------
ekspresi baru setelah direduksi : 1+2*e/f^g-h*j
jumlah tanda 6

ekspresi : 1+2*e/f^g-h*j
tanda ke : 2 yang mau dikerjakan *
batas kiri : 3 batas kanan : 5
karakter batas kiri : 2 karakter batas kanan : e
--------- substitusi x3=x2*e -----------
ekspresi baru setelah direduksi : 1+3/f^g-h*j
jumlah tanda 5

ekspresi : 1+3/f^g-h*j
tanda ke : 3 yang mau dikerjakan ^
batas kiri : 5 batas kanan : 7
karakter batas kiri : f karakter batas kanan : g
--------- substitusi x4=f^g -----------
ekspresi baru setelah direduksi : 1+3/4-h*j
jumlah tanda 4

ekspresi : 1+3/4-h*j
tanda ke : 2 yang mau dikerjakan /
batas kiri : 3 batas kanan : 5
karakter batas kiri : 3 karakter batas kanan : 4
--------- substitusi x5=x3/x4 -----------
ekspresi baru setelah direduksi : 1+5-h*j
jumlah tanda 3

ekspresi : 1+5-h*j
tanda ke : 1 yang mau dikerjakan +
batas kiri : 1 batas kanan : 3
karakter batas kiri : 1 karakter batas kanan : 5
--------- substitusi x6=x1+x5 -----------
ekspresi baru setelah direduksi : 6-h*j

jumlah tanda 2
ekspresi : 6-h*j
tanda ke : 2 yang mau dikerjakan *
batas kiri : 3 batas kanan : 5
karakter batas kiri : h karakter batas kanan : j
--------- substitusi x7=h*j -----------
ekspresi baru setelah direduksi : 6-7
jumlah tanda 1
--------- x8=x6-x7
\end{verbatim}

\section{MULTIPALINDROM}
\par
\indent
\indent
Palindrom adalah kata yang dapat dibaca sama saja baik dari kiri ke kanan
ataupun dari kanan ke kiri. Suatu palindrom sedikitnya berisi satu huruf.
Misalnya, "malam", "a" dan "ada" masing-masing adalah palindrom. Sebaliknya,
setiap kata bukan merupakan palindrom dapat dianggap sebagai deretan sejumlah
palindrom. Dengan kata lain, kata tersebut dapat dipecah-pecahkan ke dalam
sejumlah palindrom. Jadi, setiap kata pada dasarnya dapat dipandang sebagai
multipalindrom yang tersusun atas n palindrom, dengan $n > 0$. Untuk setiap
kata terdapat sejumlah kemungkinan harga n. Dengan definisi itu maka setiap
palindrom adalah multipalindrom dengan jumlah minimal n=1. Misalnya, kata
"minimisasi" terdiri atas sedikitnya 2 palindrom yaitu "minim"-"isasi"
(Red : ralat dan versi sebelumnya).
\par
\indent
\indent
Buatlah suatu program dengan nama MULTIPAL.AS yang akan menghitung jumlah
palindrom minimal dari suatu kata yang diberikan.
\subsection{Format Masukan}
\par
\indent
\indent
File masukan adalah MULTIPAL.IN yang hanya berisi kata untuk dipecah-pecah ke
dalam sejumlah palindrom. Karakter-karakter untuk membentuk kata adalah huruf
kecil (a-z). Panjang dari kata tidak akan lebih dari 100 huruf.
\subsection{Format Keluaran}
\par
\indent
\indent
Keluaran dituliskan dalam file MULTIPAL.OUT yang menyebutkan jumlah terkecil
palindrom yang dapat dibuat. \\

Contoh-contoh : \\

\begin{verbatim}
MULTIPAL.IN     MULTIPAL.IN      MULTIPAL.IN

anaban          abaccbcb         anavolimilana

MULTIPAL.OUT    MULTIPAL.OUT     MULTIPAL.OUT

2               3                5

PENJELASAN CONTOH :
#1 a naban
#2 aba cc bcb
#3 ana v o limil ana

\end{verbatim}
\subsection{Contoh Penyelesaian}
\begin{verbatim}

program mencari_multipaliandrom_pada_suatu_tulisan;
{ dikembangkan oleh dwi sakethi
                    dwijim@unila.ac.id
                    dwijim@gmail.com
		    0816 403 432
  pada tanggal 18 agustus 2006 }

uses crt;

var file_data,file_hasil : text;
    tulisan_selesai,tulisan_asli,potongan_tulisan,tulisan : string;
    selesai : boolean;
    jumlah_paliandrom,panjang_tulisan_tetap,paliandrom : byte;
    mulai_tulisan_baru,jumlah_looping,batas_kanan : byte;

{
  fungsi ini memberikan nilai 1 jika kata yang dicek berupa
  paliandrom, jika kata yg dicek bukan merupakan paliandrom
  maka fungsi ini memberikan nilai 0
}
function cek_paliandrom(tulisan_yg_dicek : string):byte;
var hasil                : byte;
    ii,panjang_tulisan_x : byte;
    hasil_reverse        : string;
begin
  hasil             := 0;
  hasil_reverse     := '';
  panjang_tulisan_x := length(tulisan_yg_dicek);
  for ii:=panjang_tulisan_x downto 1 do
      begin
        hasil_reverse := hasil_reverse + 
                         copy (tulisan_yg_dicek,ii,1);
      end;
  if tulisan_yg_dicek=hasil_reverse then
     hasil:=1;
  cek_paliandrom := hasil;
end;

{ --- program utama --- }
begin
   
   clrscr;
   { membuka file data }
   assign(file_data,'multipal.in');
   reset(file_data);

   { membaca data tulisan }
   read(file_data,tulisan);
   writeln('tulisan asal : ',tulisan);

   { membuat file hasil }
   assign(file_hasil,'multipal.out');
   rewrite(file_hasil);

   { proses pencarian paliandrom dilakukan sampai batas
     akhir tulisan }
   selesai               := false;
   batas_kanan           := length(tulisan);
   panjang_tulisan_tetap := length(tulisan);
   potongan_tulisan      := tulisan;
   tulisan_asli          := tulisan;
   tulisan_selesai       := '';
   jumlah_paliandrom     := 0;
   repeat
     paliandrom :=cek_paliandrom(potongan_tulisan);
     writeln('asal : ',potongan_tulisan, ' cek : ',paliandrom);
     if paliandrom=0 then
        begin
           batas_kanan := batas_kanan - 1;
           potongan_tulisan := copy(potongan_tulisan,1,batas_kanan);
        end
     else
        begin
           writeln('paliandrom : ',potongan_tulisan);
           tulisan_selesai := tulisan_selesai + potongan_tulisan;
           mulai_tulisan_baru := length(potongan_tulisan)+1;
           potongan_tulisan := copy(tulisan,mulai_tulisan_baru,panjang_tulisan_tetap-mulai_tulisan_baru+1);
           panjang_tulisan_tetap := length(potongan_tulisan);
           batas_kanan := length(potongan_tulisan);
           tulisan := potongan_tulisan;
           if tulisan<>'' then
              jumlah_paliandrom := jumlah_paliandrom + 1;
           readln;
        end;
     writeln('tulisan kontrol : ',tulisan_selesai);
     if tulisan_selesai=tulisan_asli then
        selesai := true;
   until selesai;

   writeln('jumlah paliandrom : ',jumlah_paliandrom);
   readln;

   { tulisan hasil ke file output }
   writeln(file_hasil,jumlah_paliandrom);
   { menutup kembali file yg telah diakses }
   close(file_hasil);
   close(file_data);
end.
\end{verbatim}
\subsection{Hasil Program}
\begin{verbatim}
cek : 0
tulisan kontrol :
asal : anavolimilana cek : 0
tulisan kontrol :
asal : anavolimilan cek : 0
tulisan kontrol :
asal : anavolimila cek : 0
tulisan kontrol :
asal : anavolimil cek : 0
tulisan kontrol :
asal : anavolimi cek : 0
tulisan kontrol :
asal : anavolim cek : 0
tulisan kontrol :
asal : anavoli cek : 0
tulisan kontrol :
asal : anavol cek : 0
tulisan kontrol :
asal : anavo cek : 0
tulisan kontrol :
asal : anav cek : 0
tulisan kontrol :
asal : ana cek : 1
paliandrom : ana

tulisan kontrol : ana
asal : volimilana
cek : 0
tulisan kontrol : ana
asal : volimilana cek : 0
tulisan kontrol : ana
asal : volimilan cek : 0
tulisan kontrol : ana
asal : volimila cek : 0
tulisan kontrol : ana
asal : volimil cek : 0
tulisan kontrol : ana
asal : volimi cek : 0
tulisan kontrol : ana
asal : volim cek : 0
tulisan kontrol : ana
asal : voli cek : 0
tulisan kontrol : ana
asal : vol cek : 0
tulisan kontrol : ana
asal : vo cek : 0
tulisan kontrol : ana
asal : v cek : 1
paliandrom : v

tulisan kontrol : anav
asal : olimilana

cek : 0
tulisan kontrol : anav
asal : olimilana cek : 0
tulisan kontrol : anav
asal : olimilan cek : 0
tulisan kontrol : anav
asal : olimila cek : 0
tulisan kontrol : anav
asal : olimil cek : 0
tulisan kontrol : anav
asal : olimi cek : 0
tulisan kontrol : anav
asal : olim cek : 0
tulisan kontrol : anav
asal : oli cek : 0
tulisan kontrol : anav
asal : ol cek : 0
tulisan kontrol : anav
asal : o cek : 1
paliandrom : o

tulisan kontrol : anavo
asal : limilana

cek : 0
tulisan kontrol : anavo
asal : limilana cek : 0
tulisan kontrol : anavo
asal : limilan cek : 0
tulisan kontrol : anavo
asal : limila cek : 0
tulisan kontrol : anavo
asal : limil cek : 1
paliandrom : limil

tulisan kontrol : anavolimil
asal : ana

cek : 0
tulisan kontrol : anavolimil
asal : ana cek : 1
paliandrom : ana

tulisan kontrol : anavolimilana
asal :

cek : 1
paliandrom :

tulisan kontrol : anavolimilana

jumlah paliandrom : 5

\end{verbatim}
\section{Menghitung Jumlah Huruf}
\par
\indent
\indent
Masalah mencari jumlah huruf pada suatu kata atau kalimat. Soal 
yang lebih jelas, mudah-mudahan kapan-kapan akan ditulis di sini.

\subsection{Contoh Penyelesain}
\par
\indent
\indent
Contoh penyelesaiannya seperti berikut :
\begin{verbatim}
{ -----------------------------------------------------
  program untuk menghitung jumlah huruf dan jenisnya
  dibuat dengan bahasa pascal
  diproses dengan sistem operasiGNU Linux Ubuntu
  compiler free pascal 
  dwi sakethi http://dwijim.staff.unila.ac.id

  nama file : hitung-jumlah-huruf.pas

  ----------------------------------------------------- }

uses crt;
{ karena ada perintah cetak ke layar }

var kalimat : string;
    huruf_ke, huruf_ke_isi : byte;
    jumlah_huruf_ke : array[1..100] of byte;
    isi_huruf_ke    : array [1..100] of string;

{ prosedur mencetak identitas pembuat program }
procedure identitas_pembuat;
begin
  textcolor(yellow+blink);
  gotoxy(1,24);write('dwi sakethi http://dwijim.staff.unila.ac.id');
  textcolor(white);
end;

{ menentukan ke mana ular akan bergerak }
procedure masukan_kalimat;
begin
  gotoxy(1,1);write('masukan kalimatnya :');
  readln(kalimat);
end;

{ menentukan ke mana ular akan bergerak }
procedure hitung_huruf;
var panjang_kalimat : byte;
    huruf_sekarang : string;
    jumlah_huruf_yang_ada : byte;
    ada_huruf : boolean;
begin
  jumlah_huruf_yang_ada := 1;
  panjang_kalimat := length(kalimat);
  writeln('panjang kalimat : ',panjang_kalimat);
  for huruf_ke:=1 to panjang_kalimat do
    begin
      huruf_sekarang :=copy(kalimat,huruf_ke,1); 
      { dari kata atau kalimat diambil per huruf selain spasi}

      if huruf_sekarang <> ' ' then
       begin
         ada_huruf := false;
         for huruf_ke_isi:=1 to jumlah_huruf_yang_ada do
          begin
             if huruf_ke = 1 then 
                begin
                 { buat array yang berisi huruf-huruf yang ditemukan
                   sampai dengan proses ini,
                   untuk huruf pertama, pasti jadi elemen pertama }
                 isi_huruf_ke[1] := huruf_sekarang;
                 jumlah_huruf_ke[1] := 1;
                 ada_huruf := true; 
                end
             else
                begin
                 if isi_huruf_ke[huruf_ke_isi] = huruf_sekarang then
                    begin
                       inc(jumlah_huruf_ke[huruf_ke_isi]);
                       ada_huruf := true;
                    end;
                end
          end; { akhir looping for huruf_ke_isi }
            
          { jika huruf yang sedang diproses tidak ada di antara salah
            satu dari array huruf yang sudah ada maka
            ini berarti huruf baru dan jumlahnya pasti 1 }
          if (ada_huruf = false)  then
             begin
                inc(jumlah_huruf_yang_ada);
                isi_huruf_ke[jumlah_huruf_yang_ada]    := huruf_sekarang;
                jumlah_huruf_ke[jumlah_huruf_yang_ada] := 1;
             end; { akhir if ada_huruf = false }
       end; { akhir if huruf_sekarang }
    end; { akhir looping kalimat }

  for huruf_ke:=1 to jumlah_huruf_yang_ada do
    begin
      writeln(isi_huruf_ke[huruf_ke],' : ',jumlah_huruf_ke[huruf_ke]);
    end;
end;

{ ---------- program utama ----------- }
begin
  clrscr;
  identitas_pembuat;
  masukan_kalimat;
  hitung_huruf;
end.
\end{verbatim}
\subsection{Contoh Keluaran}
\par
\indent
\indent
Contoh hasilnya seperti berikut :
\begin{verbatim}
dwijim@dwijim-desktop:~/Documents/olimpiade$ ppc386 hitung-jumlah-huruf.pas 
Free Pascal Compiler version 2.2.2-8 [2009/01/08] for i386
Copyright (c) 1993-2008 by Florian Klaempfl
Target OS: Linux for i386
Compiling hitung-jumlah-huruf.pas
Linking hitung-jumlah-huruf
99 lines compiled, 0.1 sec 
dwijim@dwijim-desktop:~/Documents/olimpiade$ 
masukan kalimatnya :buku tamu
panjang kalimat : 9
b : 1
u : 3
k : 1
t : 1
a : 1
m : 1
dwijim@dwijim-desktop:~/Documents/olimpiade$ 
\end{verbatim}
\section{Variasi Masalah Bilangan Prima}
\par
\indent
\indent
Soal ini hanya merupakan variasi dari masalah mencari bilangan
prima. Jadi jika Anda sudah bisa mencari bilangan prima maka
ini bisa menjadi latihan berikutnya.
\subsection{Contoh Penyelesain}
\par
\indent
\indent
Contoh penyelesaiannya seperti berikut :
\begin{verbatim}
{ -----------------------------------------------------
  program untuk mencetak bilangan prima sebanyak n buah
  dan merupakan bilangan prima ke-i dari deretan 
  bilangan prima
  dibuat dengan bahasa pascal
  diproses dengan sistem operasiGNU Linux Ubuntu
  compiler free pascal 
  dwi sakethi http://dwijim.staff.unila.ac.id
  
  nama file : prima-k.pas
  ----------------------------------------------------- }

uses crt;
{ karena ada perintah cetak ke layar }

var banyaknya_bilangan : byte;
    bilangan_prima_ke  : array[1..100] of byte;
    bilangan_prima     : longint;
    ketemu_prima       : longint;

{ prosedur mencetak identitas pembuat program }
procedure identitas_pembuat;
begin
  textcolor(yellow+blink);
  gotoxy(1,24);write('dwi sakethi http://dwijim.staff.unila.ac.id');
  textcolor(white);
end;

{ memasukkan data banyaknya bilangan prima yang dicari
  dan urutan masing-masing bilangan prima }
procedure masukan_data;
var bilangan_ke : byte;
begin
 gotoxy(1,1);write('Berapa banyak bilangan prima :');
 readln(banyaknya_bilangan);
 for bilangan_ke:=1 to banyaknya_bilangan do
 begin
   write('Bilangan prima ke : ');
   readln(bilangan_prima_ke[bilangan_ke]);
 end;
end;

{ mencek apakah suatu bilangan termasuk bilangan prima atau bukan}
function cek_prima_apa_bukan(bilangan_ini:longint):boolean;
var bilangan_sekarang : longint;
begin
  cek_prima_apa_bukan:=TRUE;
  for bilangan_sekarang:=2 to bilangan_ini-1 do
    begin
      if (bilangan_ini mod bilangan_sekarang)=0 then 
         begin
           cek_prima_apa_bukan:=FALSE;
           exit;
         end;
    end;
end;

{ ---------- program utama ----------- }
{ mencetak bilangan prima }
procedure mencetak_hasil;
var bilangan_ke : byte;
    urutan      : byte;
begin
 for bilangan_ke:=1 to banyaknya_bilangan do
 begin
   write('Bilangan prima ke : ',bilangan_prima_ke[bilangan_ke],' adalah : ');
   urutan := 0;
   bilangan_prima := 2;
   repeat
     if cek_prima_apa_bukan(bilangan_prima)=TRUE then
        begin
           inc(urutan); 
           ketemu_prima := bilangan_prima;
           { jika suatu bilangan termasuk bilangan prima maka
             urutan bertambah yang tadinya 0 menjadi 1 dst
             kemudian ditandai juga bahwa bilangan itu adalah 
             bilangan prima }
        end;
     inc(bilangan_prima);
   until urutan=bilangan_prima_ke[bilangan_ke];
   writeln(ketemu_prima);
 end;
end;

begin
  clrscr;
  identitas_pembuat;
  masukan_data;
  mencetak_hasil;
  writeln;
end.
\end{verbatim}
\subsection{Contoh Keluaran}
\par
\indent
\indent
Contoh hasilnya seperti berikut :
\begin{verbatim}
dwijim@dwijim-desktop:~/Documents/olimpiade$ ppc386 prima-k.pas 
Free Pascal Compiler version 2.2.2-8 [2009/01/08] for i386
Copyright (c) 1993-2008 by Florian Klaempfl
Target OS: Linux for i386
Compiling prima-k.pas
Linking prima-k
92 lines compiled, 0.1 sec 
dwijim@dwijim-desktop:~/Documents/olimpiade$ 
Berapa banyak bilangan prima :3
Bilangan prima ke : 4
Bilangan prima ke : 1
Bilangan prima ke : 8
Bilangan prima ke : 4 adalah : 7
Bilangan prima ke : 1 adalah : 2
Bilangan prima ke : 8 adalah : 19

dwijim@dwijim-desktop:~/Documents/olimpiade$ 
\end{verbatim}
\end{document}




----------------------------------------------------------------------
Massa

Suatu molekul terdiri atas sejumlah atom dan tersusun membentuk rumus kimia yang
dituliskan dengan huruf-huruf yang menyatakan masing-masing atom ini. Misalnya
H menyatakan atom hidrogen, C menyatakan atom karbon, O menyatakan atom
oksigen. Jadi rumus kimia COOH menyatakan suatu molekul yang berisikan satu
atom karbon, dua atom oksigen dan satu atom hidrogen.

Untuk menuliskan rumus ini secara efisien  kita menggunakan aturan-aturan
berikut ini.

Huruf-huruf yang menyatakan beberapa atom dapat dikelompokkan dengan pembatas
tanda kurung yang disebut juga dengan istilah gugus atom. Misalnya rumus CH(OH)
berisi gugus OH. Dalam suatu gugus bisa terdapat gugus-gugus lebih kecil. Untuk
menyederhanakan suatu rumus kimia, kemunculan sejumlah huruf secara
berturut-turut dapat digantikan dengan satu huruf saja tapi diikuti oleh suatu
bilangan yang menyebutkan jumlah kemunculannya. Misalnya huruf COOHHH dapat
ditulis sebagai CO2H3 dan ia mempresentasikan suatu molekul yang berisikan
satu atom karbon, dua atom oksigen dan tiga atom hidrogen.


Selanjutnya, kemunculanya yang berturut-turut dari gugus yang sama  dapat
digantikan  dengan gugus tersebut diikuti oleh bilangan yang menyatakan jumlah
kemunculan gugus tersebut. Misalnya CH (CO2H)(CO2H)(CO2H) dapat dituliskan
sebagai CH(CO2H)3 dan molekul tersebut berisikan empat atom karbon, dua atom
oksigen dan tiga atom hidrogen.

Dalam rumus kimia sebenarnya tentu bilangan yang menyatakan pengulangan
kemunculan  suatu/gugus  atom tersebut bisa berharga berapeoun asal < 1.
Dalam soal disini bilangan tersebut  dibatasi sampai dengan 9.

Massa dari suatu molekul adalah jumlah dari massa  dari setiap atom yang
tergantung didalamnya. Satu atom hidrogen memiliki massa satu, satu atom
karbon memiliki  massa 12 dan satu atom oksigen memiliki  massa 16.

Tuliskan suatu program  dengan nama MASSA.PAS yang dapat  manghitung massa
molekul dari rumus molekul yang diberikan.

Format Masukan

File mmasukan adalah file teks dengan nama MASSA.IN. File berisi satu baris
yang didalamnya tertuliskan rumuus molekul yang hendak dihitung massanya.Rumus
molekul hanya akan berisikan kemungkinan karakter-karakter H, C, O, (,), 2,
3, ......, 9.Panjang string tidak akan lebih dari 100 karakter.

Format Keluaran

File keluaran adalah file teks dengan nama MASSA.OUT. Satu-satunya baris keluaran
hanya berisikan massa dari molekul yang dinyatakan dengan rumus yang diberikan.
Bilangan massa tidak akan akan lebih besar dari 1000 karakter.

Latihan 3:Ekspresi Aljabar
Buatlah program EKSPRESI.PAS sebagai latihan ketiga anda menjelang ON.
Latihan ini mulai agak sulit. Tujuan latihan ini untuk anda membiasakan diri
dengan kompiler Free Pascal yang digunakan di webserver saat menguji perkerjaan
anda yang mungkin berbeda dengan kompiler yang sering anda gunakan selama ini.
Selain itu anda mulai berlatih pemrograman dengan tingkat kesulitan mulai
mendekati soal-soal di ON nanti.

Program anda harus dapat membaca string masukan yang berisi ekspresi
aritmetika yang terdiri atas operator pangkat-kali-bagi-tambah-kurang dan
menuliskan urutan pengejaannya yang benar. Misalnya

                                a-b+c/d*e/f^g-h*j
Untuk menentukan urutan pengerjaannya dalam penulisannya operator-operator
tersebut diberikan tingkat prioritas; pangkat laing tinggi, pada prioritas
yang sama, dan terakhir tambah dan kurang,pada prioritas yang sama. (Note:
Dalam latihan ini tanda kurung atau operator lain belum diikutsertakan).
Dengan adanya tingkat prioritas inimaka f^g harus dikerjakan sebelum e/f
atau g-h. Jika prioritas sama sehingga mana yang disebelah kiri akan dikerjakan
lebih dahulu dari yang disebelah kanan. Untuk contoh di atas c/d dikerjakan
terlebih dahulu dari pada d*e. Dengan menggunakan nama variabel sementara xi
untuk menerima hasil pengerjaan suatu operasi, maka salah satu urutan
pengerjaan ekspresi tersebut adalah:xl=a-b
x2=c/d
x3=x2*e
x4=f^g
x5=x3/x4
x6=x1+x5
x7=h*j
x8=x6-x7


Masukan

Program itu harus membaca masukan dari file bernama EKSPRESI.IN. File ini akan
berisikan satu baris teks ekspresi aritmetika dengan panjang < 256 karakter.
Operator pangkat ditulis dengan simbol '^', operator kali dengan simbol '*',
operator bagi dengan simbol '/', operator tambah dengan simbol '+', dan operator
kurang dengan simbol '-'.Operand-operand-nya sendiri adalah menggunakan
karakter huruf tunggal (a-Z, A-Z) untuk memudahkan anda membaca masukan. Dalam
ekspresi tidak ada karakter spasi atau karakter lainnya selain huruf atau
karakter simbol operator tersebut di atas.

Keluaran

Program harus menuliskan keluaran dalam file bernama EKSPRESI.OUT.Keluaran
berisikan baris-baris operasi untuk mengerjakan ekspresi masukan yang dibantu
oleh variabel-variabel sementara xi. Agar keluaran menjadi unik maka urutan
sedapat mungkin dari kiri ke kanan ekspresi kecuali kalau terkait dengan
prioritas. Misalnya a-b harus ditulis lebih dahulu dari c/d karena a-b tidak
bergantung hasil c/d. Variabel-variabel sementara xi dituliskan sebagai
karakter x dan bilangan i dengan i membesar dari baris pertama ke baris
terakir.

Contoh 1

EKSPRESI.IN
a-b+c/d

FIle.OUT
x1=a-b
x2=c/d
x3=x1+x2

Contoh 2

EKSPRESI.IN
c/d*e/f^g


EKSPRESI.OUT
x1=c/d
x2=x1*e
x3=f^g
x4=x2/x3

Contoh 3

EKSPRESI.IN
a-b+c/d*e/f^g-h*j

EKSPRESI.OUT
x1=a-b
x2=c/d
x3=x2*e
x4=f^g
x5=x3/x4
x6=x1+x2
x7=h*j
x8=x6-x7


Massa

Suatu molekul terdiri atas sejumlah atom dan tersusun membentuk rumus kimia yang
dituliskan dengan huruf-huruf yang menyatakan masing-masing atom ini. Misalnya
H menyatakan atom hidrogen, C menyatakan atom karbon, O menyatakan atom
oksigen. Jadi rumus kimia COOH menyatakan suatu molekul yang berisikan satu
atom karbon, dua atom oksigen dan satu atom hidrogen.

Untuk menuliskan rumus ini secara efisien  kita menggunakan aturan-aturan
berikut ini.

Huruf-huruf yang menyatakan beberapa atom dapat dikelompokkan dengan pembatas
tanda kurung yang disebut juga dengan istilah gugus atom. Misalnya rumus CH(OH)
berisi gugus OH. Dalam suatu gugus bisa terdapat gugus-gugus lebih kecil. Untuk
menyederhanakan suatu rumus kimia, kemunculan sejumlah huruf secara
berturut-turut dapat digantikan dengan satu huruf saja tapi diikuti oleh suatu
bilangan yang menyebutkan jumlah kemunculannya. Misalnya huruf COOHHH dapat
ditulis sebagai CO2H3 dan ia mempresentasikan suatu molekul yang berisikan
satu atom karbon, dua atom oksigen dan tiga atom hidrogen.


Selanjutnya, kemunculanya yang berturut-turut dari gugus yang sama  dapat
digantikan  dengan gugus tersebut diikuti oleh bilangan yang menyatakan jumlah
kemunculan gugus tersebut. Misalnya CH (CO2H)(CO2H)(CO2H) dapat dituliskan
sebagai CH(CO2H)3 dan molekul tersebut berisikan empat atom karbon, dua atom
oksigen dan tiga atom hidrogen.

Dalam rumus kimia sebenarnya tentu bilangan yang menyatakan pengulangan
kemunculan  suatu/gugus  atom tersebut bisa berharga berapeoun asal < 1.
Dalam soal disini bilangan tersebut  dibatasi sampai dengan 9.

Massa dari suatu molekul adalah jumlah dari massa  dari setiap atom yang
tergantung didalamnya. Satu atom hidrogen memiliki massa satu, satu atom
karbon memiliki  massa 12 dan satu atom oksigen memiliki  massa 16.

Tuliskan suatu program  dengan nama MASSA.PAS yang dapat  manghitung massa
molekul dari rumus molekul yang diberikan.

Format Masukan

File mmasukan adalah file teks dengan nama MASSA.IN. File berisi satu baris
yang didalamnya tertuliskan rumuus molekul yang hendak dihitung massanya.Rumus
molekul hanya akan berisikan kemungkinan karakter-karakter H, C, O, (,), 2,
3, ......, 9.Panjang string tidak akan lebih dari 100 karakter.

Format Keluaran

File keluaran adalah file teks dengan nama MASSA.OUT. Satu-satunya baris
keluaran hanya berisikan massa dari molekul yang dinyatakan dengan rumus yang
diberikan.Bilangan massa tidak akan akan lebih besar dari 10000.

CONTOH SOAL -SOAL PASCAL

Latihan 1:Reverse

Buatlah program REVERSE.PAS menurut penjelasan berikut ini. Sebagai latihan,
anda belajar menentukan cara bagaimana membaca masukan string yang sangat
panjang.

Masukan

Progran itu harus membaca masukan dari file bernama REVERSE.IN. File ini akan
berisikan satu baris teks dengan panjang<1000 karakter,(Hint:ini lebih panjang
dari panjang maksimum di Pascal).

Keluaran

Program harus menuliskan keluaran dalam file REVERSE.OUT.Keluaran adalah hanya
satu baris teks yaitu string hasil pembalikan yang telah anda lakukan dari string
masukan.




